\documentclass{beamer}
\usepackage[utf8]{inputenc}
\usepackage{hyperref}
\usepackage{multicol}
\usepackage{hyperref}
\usepackage{amsmath}
\usepackage[english]{babel}
\usepackage{algorithm}
\usepackage[noend]{algpseudocode}

\inputencoding{utf8}

\mode<presentation> {
    \usetheme{Madrid}
}

\usepackage{graphicx}
\usepackage{booktabs}

\title[Heap]{Heap Sort}
\author{Ernesto Rodriguez - Juan Roberto Alvaro Saravia}
\institute{
    Universidad Francisco Marroquin \\
    \medskip \textit{ernestorodriguez@ufm.edu - juanalvarado@ufm.edu}
}

\date[\today]{}

\begin{document}

\begin{frame}
\titlepage
\end{frame}

\newcommand{\buildHeap}[2]{
    
\begin{frame}

\frametitle{Crear Heap}

\begin{algorithm}[H]
    \caption{Crear un heap}
    \begin{algorithmic}[1]
    \Procedure{crear\_heap}{$Ns$}

    \For{$i\gets\mathtt{len}(Ns)/2$ {\bf downto} 0}
    \State $\mathtt{heapify}(Ns,i)$
    \EndFor
    \EndProcedure
    \end{algorithmic}
\end{algorithm}

\begin{itemize}
    \item{¿Que invariante se utilizara? {#1}}
    \item{¿Cual es la complejidad en tiempo? {#2}}
\end{itemize}
\end{frame}
}

\buildHeap{}{}

\newcommand{\buildHeapImg}[1]{
\begin{frame}
    \frametitle{Crear Heap}
    \begin{center}
    \includegraphics[width=12cm]{#1}
    \end{center}
\end{frame}
}

\buildHeapImg{crearHeap.png}
\buildHeapImg{crearHeap2.png}
\buildHeapImg{crearHeap3.png}

\buildHeap{
    \begin{itemize}
        \item{Los indices $2i$ e $2i+1$ encabezan un heap}
    \end{itemize}
}{}

\buildHeap{
    \begin{itemize}
        \item{Los indices $2i$ e $2i+1$ encabezan un heap}
    \end{itemize}
}{
    \begin{itemize}
        \item{$\mathcal{O}(nlog(n))$}
    \end{itemize}
}

\buildHeap{
    \begin{itemize}
        \item{Los indices $2i$ e $2i+1$ encabezan un heap}
    \end{itemize}
}{
    \begin{itemize}
        \item{$\mathcal{O}(nlog(n))$}
        \item{¿Nos podemos acercar m\'as?}
    \end{itemize}
}



\end{document}