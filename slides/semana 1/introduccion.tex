\documentclass{beamer}
\usepackage[utf8]{inputenc}
\usepackage{hyperref}
\usepackage{multicol}
\usepackage{hyperref}

\inputencoding{utf8}

\mode<presentation> {
    \usetheme{Madrid}
}

\usepackage{graphicx}
\usepackage{booktabs}

\title[Introducciòn]{Introducci\'on al curso Algoritmia y Complejidad}
\author{Ernesto Rodriguez - Juan Roberto Alvaro Saravia}
\institute{
    Universidad Francisco Marroquin \\
    \medskip \textit{ernestorodriguez@ufm.edu - juanalvarado@ufm.edu}
}

\date[\today]{}

\begin{document}

\begin{frame}
\titlepage
\end{frame}

\begin{frame}
    \frametitle{Objetivos del curso}
    \begin{itemize}
        \item{Familiarizarse con varios algoritmos que se
        utilizan en las ciencias de la computaci\'on}
        \item{Aprender a analizar problemas con el proposito
        de poder aplicar algoritmos para solucionarlos}
        \item{Aprender a razonar sobre la complejidad y
        rendimiento de una soluci\'on algoritmica}
        \item{Familiarizarse con el analisis costo/beneficio
        de las diferentes opciones para resolver problemas}
        \item{Introducir al estudiante a la te\'oria de
        computabilidad y complejidad}
        \item{Familiarizar al estudiante con el concepto de
        intratabilidad y ayudarle a manejarla}
    \end{itemize}
\end{frame}

\begin{frame}
    \frametitle{Estructura del curso}
    \begin{itemize}
        \item{El curso se dividira en dos modulos:
        \begin{itemize}
            \item{Algoritmos y complejidad}
            \item{Te\'oria de la computabilidad y complejidad}
        \end{itemize}
        }
    \end{itemize}
\end{frame}

\begin{frame}
\frametitle{Algoritmos y Complejidad}
\begin{itemize}
    \item{Analisis asintotico y notaci\'on}
    \item{Familias de algoritmos en ciencias de la computaci\'on:
    \begin{itemize}
        \item{Ordenamiento}
        \item{Busqueda}
        \item{Optimizaci\'on}
        \item{Numericos}
        \item{Algoritmos aleatorizados}
    \end{itemize}
    \item{Aplicaciones de algoritmos:
    \begin{itemize}
        \item{Almacenamiento}
        \item{Optimizaci\'on de procesos}
        \item{Inteligencia artificial}
        \item{Graficos}
        \item{Seguridad}
    \end{itemize}
    }
    \item{Estos algoritmos aparecen en entrevistas muy amenudo}
    }
\end{itemize}
\end{frame}

\begin{frame}
\frametitle{Teoria de la Computabilidad y Complejidad}
\begin{itemize}
    \item{Lenguajes formales:
    \begin{itemize}
        \item{No-interesantes}
        \item{Interesantes}
        \item{Decidibles}
        \item{Enumerables}
        \item{Regulares}
    \end{itemize}
    }
    \item{Decidibilidad y el \emph{Halting Problem}}
    \item{Modelos de computaci\'on:
    \begin{itemize}
        \item{Definici\'on de computabilidad}
        \item{Maquinas de Turing}
        \item{Calculo-$\lambda$}
        \item{Funciones recursivas}
    \end{itemize}
    }
    \item{Clases de computabilidad y herarquia computable}
    \item{Clases de complejidad:
    \begin{itemize}
        \item{Lenguajes polinomiales (P)}
        \item{Lenguajes polinomiales no deterministicos (NP)}
        \item{Lenguajes exponenciales}
        \item{Por espacio y por tiempo}
    \end{itemize}
    }
\end{itemize}
\end{frame}

\begin{frame}
\frametitle{Acerca de Ernesto Rodriguez}
\begin{itemize}
    \item{Licenciatura de Jacobs University Bremen,
    enfocado en Machine Learning.}
    \item{Maestria de Utrecht University, enfocado
    en Lenguajes de Programaci\'on.}
    \item{Trabaje en varias empresas, incluyendo Microsoft.}
    \item{Dos publicaciones academicas.}
    \item{Varios proyectos open source.}
    \item{Intereses: Hackathons Cryptomonedas, Programaci\'on funcional,
    Machine Learning, Computaci\'on Teorica.}
\end{itemize}
\end{frame}

\begin{frame}
    \frametitle{Formato del curso}
    \begin{itemize}
        \item{Una hoja de trabajo (casi) semanal.}
        \item{Se utilizara Git y Github para entregar trabajos.}
        \item{Todo trabajo escrito se realizara con Latex.}
        \item{Dos ex\'amenes parciales t\'eoricos.}
        \item{Ex\'amen final.}
        \item{No es un curso de programaci\'on, pero
        se utilizara la porgramaci\'on para entender los temas.}
        \item{Se utilizara \href{https://www.python.org/}{Python 3} como lenguaje de programaci\'on}
    \end{itemize}
\end{frame}

\begin{frame}
\frametitle{Herramientas y Recursos}
\begin{multicols*}{2}
    {\bf Herramientas} \\
\begin{itemize}
    \item \href{https://code.visualstudio.com/}{Visual Studio Code}
    \item \href{https://git-scm.com/}{Git}
    \item \href{https://github.com/}{Github}
    \item \href{https://www.latex-project.org/}{Latex}
    \item \href{https://www.python.org/}{Python}
    \item \href{https://manjaro.org/}{Manjaro Linux\\
    \tiny{No importa a que campo de la computaci\'on se dediquen,
        Linux siempre estara ahi.}}
\end{itemize}
\columnbreak
{\bf Recursos}
\begin{itemize}
    \item{\href{https://github.com/netogallo/algoritmos-ufm-2018/blob/master/recursos/Programa.pdf}{Programa del Curso}}
    \item{Repositorio del curso \cite{Repositorio}}
    \item{Introduction to Algorithms \cite{Algoritmos}}
    \item Latex Wiki \cite{Latex}
    \item Git tutorial \cite{GitTutorial}
    \item{\href{https://docs.python.org/3/tutorial/index.html}{Python tutorial}}
\end{itemize}
\end{multicols*}
\end{frame}

\begin{frame}

\frametitle{Reglas}

\begin{itemize}
\item{La asistencia a clase es requisito universitario, pero
somos adultos y si preferimos aprender por su cuenta, lo acepto.
Sin embargo, recomiendo que asistan a clase.}
\item{Durante clase, respetar al profesor y a sus compa\~neros.
No interrumpir ni burlarse de otras personas.}
\item{Ser respetuoso y constructivo a la hora de criticar.}
\item{No se tolera el plagio.
\\\small{Es m\'as dificil aprender a copiar que aprender a programar.}}
\item{Ser puntuales a la hora de entregar tareas.}
\end{itemize}
\end{frame}

\begin{frame}
\frametitle{Referencias}
\bibliography{../../recursos/referencias}
\bibliographystyle{plain}
\end{frame}

\end{document}